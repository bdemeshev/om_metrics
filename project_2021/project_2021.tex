% arara: xelatex
% arara: xelatex

\documentclass[12pt]{article}

\usepackage{tikz} % картинки в tikz
\usepackage{microtype} % свешивание пунктуации

\usepackage{array} % для столбцов фиксированной ширины

\usepackage{indentfirst} % отступ в первом параграфе

\usepackage{sectsty} % для центрирования названий частей
\allsectionsfont{\centering}

\usepackage{amsmath, amssymb, amsthm} % куча стандартных математических плюшек

\usepackage{amsfonts}

\usepackage{comment}

\usepackage[top=2cm, left=1.2cm, right=1.2cm, bottom=2cm]{geometry} % размер текста на странице

\usepackage{lastpage} % чтобы узнать номер последней страницы

\usepackage{enumitem} % дополнительные плюшки для списков
%  например \begin{enumerate}[resume] позволяет продолжить нумерацию в новом списке
\usepackage{caption}


\usepackage{hyperref} % гиперссылки

\usepackage{multicol} % текст в несколько столбцов


\usepackage{fancyhdr} % весёлые колонтитулы
\pagestyle{fancy}
\lhead{Эконометрика, программа «Экономический анализ», НИУ-ВШЭ}
\rhead{Проектное задание.}

\rfoot{}
% \rfoot{\thepage/3}
\renewcommand{\headrulewidth}{0.4pt}
\renewcommand{\footrulewidth}{0.4pt}



% \usepackage{todonotes} % для вставки в документ заметок о том, что осталось сделать
% \todo{Здесь надо коэффициенты исправить}
% \missingfigure{Здесь будет Последний день Помпеи}
% \listoftodos - печатает все поставленные \todo'шки


% более красивые таблицы
\usepackage{booktabs}
% заповеди из документации:
% 1. Не используйте вертикальные линии
% 2. Не используйте двойные линии
% 3. Единицы измерения - в шапку таблицы
% 4. Не сокращайте .1 вместо 0.1
% 5. Повторяющееся значение повторяйте, а не говорите "то же"



\usepackage{fontspec}
\usepackage{polyglossia}

\setmainlanguage{russian}
\setotherlanguages{english}

% download "Linux Libertine" fonts:
% http://www.linuxlibertine.org/index.php?id=91&L=1
\setmainfont{Linux Libertine O} % or Helvetica, Arial, Cambria
% why do we need \newfontfamily:
% http://tex.stackexchange.com/questions/91507/
\newfontfamily{\cyrillicfonttt}{Linux Libertine O}

\AddEnumerateCounter{\asbuk}{\russian@alph}{щ} % для списков с русскими буквами
\setlist[enumerate, 2]{label=\asbuk*),ref=\asbuk*}

%% эконометрические сокращения
\let\P\relax
\DeclareMathOperator{\P}{\mathbb{P}}
\DeclareMathOperator{\Cov}{\mathbb{C}ov}
\DeclareMathOperator{\Corr}{\mathbb{C}orr}
\DeclareMathOperator{\Var}{\mathbb{V}ar}
\DeclareMathOperator{\E}{\mathbb{E}}
\DeclareMathOperator{\tr}{trace}
\newcommand \hb{\hat{\beta}}
\newcommand \hs{\hat{\sigma}}
\newcommand \htheta{\hat{\theta}}
\newcommand \s{\sigma}
\newcommand \hy{\hat{y}}
\newcommand \hY{\hat{Y}}
\newcommand \vone{\vec{1}}
\newcommand \e{\varepsilon}
\newcommand \he{\hat{\e}}
\newcommand \z{z}
\newcommand \hVar{\widehat{\Var}}
\newcommand \hCorr{\widehat{\Corr}}
\newcommand \hCov{\widehat{\Cov}}
\newcommand \cN{\mathcal{N}}



\begin{document}

% RLMS — это набор данных по индивидам и домохозяйствам России. В этом домашнем задании вам предстоит провести своё маленькое и гордое первое исследование. 

% Взять волну рлмс. Построить графики. Построить регрессию, проинтерпретировать, построить доверительные интервалы для бет, предикативные интервалы для прогнозов. Сравнить две модели. Построить график в осях первых двух компонент. Сравнить прогнозы регрессии без регуляризации с прогнозами лассо. Найти свою прикольную статистическую закономерность в данных.

Для выполнения проекта используйте данные RLMS. Волна остается та же, что и в домашнем задании. Можно поменять используемые данные на любые другие и сформулировать другой исследовательский вопрос. 

\begin{enumerate}

%\item Запишитесь на волну RLMS по ссылке
%\url{http://tiny.cc/hse_master_ea_ha_rlms}.
%\url{https://docs.google.com/spreadsheets/d/1xq7kLAnUiZny4pvpLy0xBbFQM24EgG9YwTqkyO1chuk/edit?usp=sharing}. 

%На каждую волну может зарегистрироваться не более трёх человек. При этом разрешается выполнять работу как индивидуально, так и в группе. Скажем, на волну 22 может записаться Андрей, выполняющий работу в одиночку, и команда Бориса и Владимира, выполняющих работу вместе. 

%\item Скачайте и импортируйте данные выбранной вами волны.
%Данные RLMS доступны по ссылке %\url{https://www.hse.ru/rlms/spss}.
%\item Ознакомьтесь с доступными переменными. 
%Разумно посмотреть документацию на официальном сайте.
%Будьте осторожны, в наблюдениях много пропусков!
\item Опишите используемые данные, если приняли решение поменять исследовательский вопрос. 

\item Сформилируйте исследовательский вопрос (повторите старый или сформулируйте новый), который можно изучить с помощью ваших данных.

Достаточно сформулировать вопрос о наличии статистической связи между переменными. Стоит осознавать, что обнаружение причинно-следственных связей — очень трудная задача, браться за неё мы не требуем, хотя и не запрещаем.

В постановке вопроса в качестве зависимой переменной должна использоваться количественная переменная. Среди предикторов должно быть минимум две количественных переменных и минимум одна факторная.


 %\item Визуализируйте используемые переменные и приведите писательные статистики.

\item 
Дайте краткую характеристику используемым переменным. Приведите описательные статистики.
%Используйте гистограммы и другие способы визуального представления информации.

Изобразите ваши наблюдения в осях первых двух главных компонент, выделив их из количественных регрессоров и зависимой переменной. 

\item Оцените две регрессионных модели: простую и более общую сложную. Это могут быть как те же, модели, которые вы использовали в домашнем задании, так и новые.

Для сложной модели проинтерпретируйте коэффициенты и постройте доверительные интервалы для них. С помощью теста сделайте выбор между простой и сложной моделью. 

При проведении тестов и построении доверительных интервалов используйте робастные к гетероскедастичности методы.

\item Проведите тест Рамсея для отобранной модели. Проинтерпретируйте результат тестирования.

\item Для отобранной  модели проверьте наличие мультикооллинеарности.

\item Сравните модели с LASSO по прогнозной силе.

Поделите выборку на две части: обучающую и тестовую. На обучающей оцените простую модель, сложную модель и модель LASSO. Постройте прогнозы на тестовую часть выборки. Выберите наилучшую из моделей. 

\item Для финальной  модели и выдуманного нового наблюдения постройте прогноз и предиктивный интервал для прогноза.  

\item Проверьте наличие гетероскедастичности графически и с помощью хотя бы одного формального теста. 

\item Описанные выше пункты являются минимальным требованием. Можно сделать больше!  Удачи :)

\item Работу следует представить в виде отчёта в pdf или html формате. 
В начале работы должен идти текст с графиками, в конце работы в качестве приложения должен идти код. 
Общий объем текста (без приложений) должен составлять \textbf{не более 10 страниц}.


Дедлайн сдачи - \textbf{6 декабря 2021, 20:59}. До указанного времени файл в формате pdf должен быть загружен по ссылке
\url{https://www.dropbox.com/request/YbF0bdSMZYWalPxJqZec}.

\end{enumerate}

\newpage
Если RLMS надоел, то можно посмотреть в сторону:
\begin{itemize}
\item Russian Federal State Statistic Service, \url{https://rosstat.gov.ru/};
\item Russian Federal State Statistic Service (Сборник Регионы России), \url{https://rosstat.gov.ru/folder/210/document/13204};
\item Joint Economic and Social Data Archive HSE, \url{http://sophist.hse.ru/};
\item World Bank Open Data, \url{https://data.worldbank.org/};
\item OECD Data, \url{https://data.oecd.org/}.
\end{itemize}
%Данные обследования РМЭЗ НИУ ВШЭ





\end{document}


