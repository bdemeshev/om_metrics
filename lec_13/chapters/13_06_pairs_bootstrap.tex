% !TEX root = ../om_metrics_13.tex

\begin{frame} % название фрагмента

\videotitle{Парный бутстрэп}

\end{frame}



\begin{frame}{Краткий план:}
  \begin{itemize}[<+->]
    \item Парный бутстрэп.
    \item Практические рекомендации.
  \end{itemize}

\end{frame}


\begin{frame}{Парный бутстрэп — это просто!}

На примере модели $y_i = \beta_1 + \beta_x x_i + \beta_w w_i + u_i$.

Есть исходные наблюдения $(x_i, w_i, y_i)$, где $i \in \{1, \ldots, n\}$

\begin{enumerate}[<+->]
    \item Генерируем очередную бутстрэп-выборку $(x^*_i, w^*_i, y^*_i)$, где $i \in \{1, \ldots, n\}$.
    
    Случайно выберем $n$ наблюдений из исходной выборки с повторениями.
\item Считаем очередную бутстрэп оценку коэффициента $\hat \beta^*_{xj}$ или $t$-статистику 
$t^*_{j} = (\hat \beta^*_{xj} - \hat\beta_x)/se(\hat \beta^*_{xj})$.
    \item Повторим первые два шага много раз: $j = 1, \ldots, 10000$.
\end{enumerate}
\end{frame}

\begin{frame}{Доверительный интервал}

\begin{block}{Наивный вариант}
\[
\qL(\hat \beta_x^*) \leq \beta_x \leq \qR(\hat \beta_x^*)
\]
\end{block}

\begin{block}{Вариант с $t$-статистикой}
Находим $\beta_x$ из неравенства
\[
\qL(t^*) \leq \frac{\hat \beta_x - \beta_x}{se(\hat \beta_x)} \leq \qR(t^*)    
\]
Получаем 
\[
\hat \beta_x  - se(\hat \beta_x) \qR(t^*) \leq \beta_x \leq \hat \beta_x - se(\hat \beta_x) \qL(t^*)      
\]
\end{block}

\end{frame}





\begin{frame}{Бутстрэп: рекомендации}

\alert{Общие}
\begin{enumerate}[<+->]
    \item Используйте бутстрэп!
    \item Берите большое количество (10000) бутстрэп выборок. 
\end{enumerate}

\pause
Бутстрэп — идея, а не конкретный метод. Какой выбрать?

\pause
\alert{Без регрессоров}

\begin{enumerate}[<+->]
    \item Смело берите бутстрэп $t$-статистики.
    \item Если формулы для стандартных ошибок нет, попробуйте наивный бутстрэп или бутстрэп в бутстрэпе.
\end{enumerate}    

\pause
\alert{С регрессорами}
\begin{enumerate}[<+->]
    \item Смело берите дикий бутстрэп $t$-статистики.
    \item Если матрица регрессоров $X$ не фиксирована, попробуйте парный бутстрэп.
\end{enumerate}    

\end{frame}




\begin{frame}{Источники мудрости}

\begin{enumerate}[<+->]
    \item Tim Hestenberg, What Teachers Should Know about the Bootstrap.
    \item James MacKinnon, Bootstrap Methods in Econometrics.
\end{enumerate}

\end{frame}


\begin{frame}{Резюме: бутстрэп до регрессии}


\begin{itemize}[<+->]
\item Бутстрэп: оценка распределения вместо теорем.
\item Наивный бутстрэп: сгенерируем много значений величины $\hat m^*_j$. 
\item Бутстрэп $t$-статистики: сгенерируем много значений 
\[
t^*_j = \frac{\hat m^*_j - \hat m}{se(m^*_j)}
\]
\item Бутстрэп в бутстрэпе: способ получить $se(m^*_j)$, если нет явной формулы.
\end{itemize}


\end{frame}
    



\begin{frame}{Резюме: бутстрэп и регрессия}


\begin{itemize}[<+->]
\item Параметрический бутстрэп: 
\[
    y_i^* = \hat \beta_1 + \hat\beta_x x_i + \hat\beta_w w_i + u_i^*, \quad u_i^* \sim \cN(0;\hat\sigma^2)
\]
\item Дикий бутстрэп: 
\[
    y_i^* = \hat \beta_1 + \hat\beta_x x_i + \hat\beta_w w_i + \hat u_i^{sc} v_i^*, \quad v_i^* \in \{-1, +1\}
\]
\item Парный бутстрэп: выбираем случайные наблюдения с повторениями.
\end{itemize}

\pause

\alert{Следующая лекция:} причинно-следственные связи.

\end{frame}
    