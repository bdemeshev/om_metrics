% !TEX root = ../om_metrics_13.tex

\begin{frame} % название фрагмента

\videotitle{Бутстрэп $t$-статистики}

\end{frame}



\begin{frame}{Краткий план:}
  \begin{itemize}[<+->]
    \item Бутстрэп $t$-статистики.
    \item Сравнение с наивным бутстрэпом.
  \end{itemize}

\end{frame}



\begin{frame}{Задача оценивания вероятности}

\begin{block}{Доверительный интервал для вероятности $p=\P(y_i > 0)$}
Есть случайная выборка $y_1$, \ldots, $y_n$ из непрерывного распределение, $n$ велико. 

\pause
Нашли выборочную долю положительных наблюдений $\hat p$. 

\pause
Теория говорит, что $\Var(\hat p) = \frac{p(1-p)}{n}$. 

\pause
Нашли стандартную ошибку $se(\hat p) = \sqrt{ \frac{\hat p (1 - \hat p)}{n} }$.
\end{block}

\end{frame}


\begin{frame}{Бутстрэп $t$-статистики}

\begin{enumerate}[<+->]
  \item Из исходной выборки $y_1$, \ldots, $y_n$ построим бутстрэп-выборку $y^*_1$, \ldots, $y^*_n$:

  Выберем случайно $n$ наблюдений с повторениями. 
  \item На базе бутстрэп-выборки посчитаем:
  \begin{itemize}
    \item очередную выборочную долю $\hat p_j^*$;
    \item её стандартную ошибку $se(\hat p_j^*)$;
    \item $t$-статистику 
    \[
    t^*_j = \frac{\hat p_j^* - \hat p}{se(\hat p_j^*)}.  
    \]
  \end{itemize}
  \item Повторим первые два шага много раз: $j = 1, \ldots, 10000$.
\end{enumerate}


\end{frame}
  


\begin{frame}{Формула доверительного интервала}

Хотим доверительный интервал для истинной вероятности $p$ и уже раздобыли 10000 бустрэп $t$-статистик $t^*_1$, \ldots, $t^*_{10000}$.
  
\pause
\begin{block}{Рецепт}
Находим $p$ из неравенства
\[
\qL(t^*) \leq \frac{\hat p - p}{se(\hat p)} \leq \qR(t^*)    
\]
Получаем 
\[
\hat p  - se(\hat p) \qR(t^*) \leq p \leq \hat p - se(\hat p) \qL(t^*)      
\]
\end{block}

\pause
\alert{Не пугайтесь минуса справа!} 

Скорее всего $\qL(t^*)$ меньше нуля. 

\end{frame}
      

\begin{frame}{Аналогия}

\begin{tabular}{@{}ll@{}}
  \toprule
Классика & Бутстрэп \\ 
\midrule
Параметр $p$ & Оценка $\hat p$ \\
\midrule
Исходная выборка & Бутстрэп выборки \\
Оценка $\hat p$ & Бутстрэп оценки $\hat p^*_j$ \\
Стандартная ошибка $se(\hat p)$ & Стандартные ошибки $se(\hat p^*_j)$  \\
Статистика $t = (\hat p - p) /se(\hat p)$ & Статистики $t^*_j = (\hat p^*_j - \hat p) /se(\hat p^*_j)$ \\
\bottomrule
\end{tabular}


\end{frame}



\begin{frame}{Сравнение с наивным бутстрэпом}

\begin{enumerate}[<+->]
  \item Любой бутстрэп лучше, чем отсутствие.
  \item Бутстрэп $t$-статистики лучше, чем наивный.
  \item Бутстрэп $t$-статистики требует формулы для $se(\hat \theta)$.
  \item В качестве $se(\hat \theta)$ можно использовать приближение. 
  \item Можно рассчитать $se(\hat \theta)$ с помощью бутстрэпа в бутстрэпе.
\end{enumerate}

\end{frame}
  
\begin{frame}{Рекомендация}

\begin{enumerate}[<+->]
  \item Используйте бутстрэп $t$-статистики.
  \item Если нет готовой формулы для $se(\hat \theta)$, придумайте приближенную — \alert{бутстрэп сам поправит}!
\end{enumerate}

\end{frame}

\begin{frame}{Пример приближения стандартной ошибки}

Хочу использовать бутстрэп $t$-статистики при построении интервала для медианы $m$.
Не знаю никакой формулы для $se(\hat m)$.

\pause \alert{Я:} Выборочная медиана примерно похожа на выборочное среднее. \pause
Возьму в $t$-статистике стандартную ошибку среднего!

\[
  t^*_j = \frac{\hat m^*_j - \hat m}{\alert{se(\bar y^*)}}, \quad se(\bar y^*) = \sqrt{ \frac{1}{n} \sum_{i=1}^n (y_i^* - \bar y^*)^2 / (n - 1)}.
\]

\pause
\alert{Бутстрэп:} 

Это неправильная формула, но я сам подправлю квантили!

\end{frame}


