% !TEX root = ../linal_lecture_02.tex

\begin{frame} % название фрагмента

\videotitle{Ранг оператора}

\end{frame}



\begin{frame}{Краткий план:}
  \begin{itemize}[<+->]
    \item Множество значений оператора;
    \item Ранг оператора.
  \end{itemize}

\end{frame}


\begin{frame}{Множество значений оператора}

Любой вектор $\bv$ представим в виде:
\[
\bv = v_1 \be_1 + v_2 \be_2 + \ldots + v_n \be_n
\]

\pause
По свойству линейности
\[
\LL \bv = v_1 \LL\be_1 + v_2 \LL\be_2 + \ldots + v_n\LL\be_n
\]

\pause
\begin{block}{Утверждение}
Множество значений оператора $\LL$ можно записать в виде линейной оболочки:
\[
\Image \LL  = \Span \{ \LL \be_1, \LL \be_2, \ldots, \LL \be_n  \}  
\]
\end{block}

\end{frame}



\begin{frame}{Ранг оператора}
\begin{block}{Определение}
    \alert{Рангом} линейного оператора $\LL$ называют размерность его образа:
    \[
      \rank \LL = \dim \Image \LL = \dim \Span \{ \LL\be_1, \LL\be_2, \ldots, \LL\be_n\}  
    \]
\end{block}
\end{frame}

\begin{frame}{Удаление компоненты вектора}

$\LL : \begin{pmatrix}
  a_1 \\
  a_2 \\
  a_3 \\
\end{pmatrix} \to 
\begin{pmatrix}
  a_1 \\
  a_3 \\
\end{pmatrix}$

\pause

Действие оператора $\LL$ на базисных векторах $\be_1$, $\be_2$ и $\be_3$:

$\LL : \begin{pmatrix}
  1 \\
  0 \\
  0 \\
\end{pmatrix} \to 
\begin{pmatrix}
1  \\
0  \\
\end{pmatrix}, \;
\LL : \begin{pmatrix}
  0 \\
  1 \\
  0 \\
\end{pmatrix} \to 
\begin{pmatrix}
0 \\
0 \\
\end{pmatrix}, \;
\LL : \begin{pmatrix}
  0 \\
  0 \\
  1 \\
\end{pmatrix} \to 
\begin{pmatrix}
0 \\
1 \\
\end{pmatrix}$

\pause

\[
\Image \LL = \Span \left\{\begin{pmatrix}
1 \\
0 \\
\end{pmatrix}, \;
\begin{pmatrix}
0 \\
0 \\
\end{pmatrix}, \;
\begin{pmatrix}
0 \\
1 \\
\end{pmatrix}
\right\}    
\]

\pause 
Базис для $\Image \LL$: $\left\{\begin{pmatrix}
1 \\
0 \\
\end{pmatrix}, \;
\begin{pmatrix}
0 \\
1 \\
\end{pmatrix} \right\}$

\[
\rank \LL = \dim \Image \LL = 2    
\]

\end{frame}

\begin{frame}{Ранг проекции}


Если оператор $\HH$ проецирует векторы на прямую $\ell$, то $\Image \HH = \Span \ba$, где $\ba$ — любой ненулевой вектор, лежащий на прямой $\ell$.

\pause

Ранг оператора проецирования на прямую равен $\rank \HH = 1$.

\pause

Ранг оператора проецирования $\HH$ равен размерности того множества, на которое проецируют.

\pause

\begin{block}{Определение}
Ранг оператора проецирования $\HH$ также называют \alert{следом оператора проецирования},
$\trace \HH = \rank \HH$.
\end{block}


\end{frame}


\begin{frame}{Ранг поворота}

Оператор $\Rot$ поворачивает плоскость на $30^{\circ}$ градусов против часовой стрелки.

\pause

Поворачивая различные векторы, можно получить любой вектор на плоскости, $\Image \Rot = \R^2$.

\pause

Базис образа: $\{ \be_1, \be_2 \}$, значит $\rank \Rot  = 2$.



\end{frame}
    


\begin{frame}{Ограничения на ранг}

\begin{block}{Утверждение}
Ранг оператора $\LL: \R^n \to \R^k$ не превосходит ни $n$, ни $k$. 
\end{block}

\pause

\begin{block}{Доказательство}
Базис во всём $\R^k$ содержит $k$ элементов, значит базис образа не больше.
\pause

Образ получается как $\Span\{\LL \be_1, \LL\be_2, \ldots, \LL \be_n \}$.
\end{block}
\end{frame}

\begin{frame}{Ранг произведения операторов}

\begin{block}{Утверждение}
Ранг произведения $\LL_1: \R^n \to \R^k$ и $\LL_2: \R^k \to \R^d$ не превосходит ранга сомножителей, $\rank (\LL_2 \LL_1) \leq \min \{\rank \LL_1, \rank \LL_2 \}$.
\end{block}

\pause
\begin{block}{Доказательство}
\[
\Image(\LL_2\LL_1) \subset \Image(\LL_2)  
\]
\pause
Если $\Image \LL_1 = \Span\{ \bv_1, \bv_2, \ldots, \bv_p \}$, то $\Image (\LL_2 \LL_1 ) =\Span\{ \LL_2 \bv_1, \LL_2 \bv_2, \ldots, \LL_2 \bv_p \}$.

\end{block}
\end{frame}
  






\begin{frame}{Ранг матрицы}

\begin{block}{Определение}
\alert{Рангом матрицы} называют ранг соответствующего оператора.
\end{block}

\pause

\begin{block}{Утверждение}
Ранг матрицы равен максимальному количеству линейно независимых столбцов матрицы. 
\end{block}

\pause
\begin{block}{Доказательство}
Именно эти линейно-независимые столбцы и будут базисом в линейной оболочке $\Image \LL$. 
\end{block}


\end{frame}




