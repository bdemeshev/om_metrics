% !TEX root = ../om_metrics_13.tex

\begin{frame} % название фрагмента

  \videotitle{Дикий бутстрэп}
  
  \end{frame}
  
  
  
  \begin{frame}{Краткий план:}
    \begin{itemize}[<+->]
      \item Добавим модель и предикторы!
      \item Вариации дикого бутстрэпа.
    \end{itemize}
  \end{frame}
  
  
  
  \begin{frame}{Постановка задачи}
  Модель $y_i = \beta_1 + \beta_x x_i + \beta_w w_i + u_i$ и \alert{сомнения} в предпосылках на распределение $u_i$. 
  
  \pause
  Применили обычный МНК и получили оценки $\hat \beta = (\hat \beta_1, \hat \beta_x, \hat \beta_w)$ и
  наивную оценку дисперсии $\hat \sigma^2_u = RSS / (n - k)$.
  
  \pause
  Хотим доверительный интервал для $\beta_x$ с корректной вероятностью накрытия. 
  \end{frame}
  
  
  
  \begin{frame}{Дикий бутстрэп}
  
    \begin{enumerate}[<+->]
      \item Для бутстрэп выборки сохраняем полностью исходную матрицу регрессоров $X$.
      \item Генерируем бутстрэп выборку ошибок:
      \[
       u^*_i \sim \alert{\ldots}
      \]
      \item Генерируем бутстрэп выборку зависимой переменной:
      \[
        y^*_i = \hat \beta_1 + \hat\beta_x x_i + \hat \beta_w w_i + u^*_i
      \]
      
      \pause
      Используем $\hat\beta$ \alert{исходной регрессии}.
  
  \item Считаем очередную бутстрэп оценку коэффициента $\hat \beta^*_{xj}$ или $t$-статистику 
  $t^*_{j} = (\hat \beta^*_{xj} - \hat\beta_x)/se(\hat \beta^*_{xj})$.
        \item Повторяем шаги два, три и четыре много раз: $j= 1, \ldots, 10000$.
  \end{enumerate}
    
  \end{frame}
  

  

\begin{frame}{Дикий бутстрэп: детали}

Генерирование бутстрэп выборки ошибок $u^*_1$, \ldots, $u^*_n$.

  \begin{enumerate}[<+->]
    \item Рассчитываем отмасштабированные остатки исходной регрессии
    \[
    \hat u_i^{sc} = \frac{\hat u_i}{\sqrt{1 - H_{ii}}}, \quad H = X(X'X)^{-1}X'.
    \]
    
  
    Это действие \alert{приравнивает} дисперсии остатков при гомоскедастичности. 
    \item Домножаем отмасштабированные ошибки на плюс или минус единицу
  \[
    u^*_i  = \hat u_i^{sc} \cdot v_i^*, \quad  v_i^* \in \{-1, +1\} \text{ равновероятно}.
  \]
\end{enumerate}
\pause
\alert{Теорема.} При гомоскедастичности ошибок $u_i$ дисперсия остатка $\hat u_i$ пропорциональна $1 - H_{ii}$.

\end{frame}




  
  \begin{frame}{Интервал: наивный вариант}
  
  
  Хотим доверительный интервал для $\beta_x$.
  
  Есть 10000 штук бутстрэп оценок $\hat \beta^*_{x,1}$, \ldots, $\hat \beta^*_{x, 10000}$.
  
  \pause
  \begin{block}{Наивный вариант}
  \[
  \qL(\hat \beta_x^*) \leq \beta_x \leq \qR(\hat \beta_x^*)
  \]
  \end{block}
  
  
  \end{frame}
  
  
  \begin{frame}{Интервал: вариант с $t$-статистикой}
  
  Хотим доверительный интервал для $\beta_x$.
  
  Есть 10000 штук бутстрэп $t$-статистик
  $t^*_{1}$, \ldots, $t^*_{10000}$.
  
  
  \pause
  \begin{block}{Вариант с $t$-статистикой}
  Находим $\beta_x$ из неравенства
  \[
  \qL(t^*) \leq \frac{\hat \beta_x - \beta_x}{se(\hat \beta_x)} \leq \qR(t^*)    
  \]
  Получаем 
  \[
  \hat \beta_x  - se(\hat \beta_x) \qR(t^*) \leq \beta_x \leq \hat \beta_x - se(\hat \beta_x) \qL(t^*)      
  \]
  \end{block}
  
  
  \end{frame}
  
    