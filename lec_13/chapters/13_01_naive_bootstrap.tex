% !TEX root = ../om_metrics_13.tex

\begin{frame} % название фрагмента

\videotitle{Наивный бутстрэп}

\end{frame}



\begin{frame}{Краткий план:}
  \begin{itemize}[<+->]
    \item Общая идея бутстрэпа.
    \item Наивный бустрэп.
  \end{itemize}

\end{frame}


\begin{frame}{Спасение утопающих}

  \alert{Бутстрэп} позволяет не думать о том, как распределены статистики!
  
  \pause

  Вы всё ещё выбираете степени свободы? 
  
  \pause
  
  Тогда мы идёт к вам! 
\end{frame}

\begin{frame}{Общая логика проверки гипотез}

\begin{enumerate}[<+->]
  \item Теорема. При верной $H_0$, \alert{идеальных условиях} и \alert{$n\to\infty$} статистика $S \to \chi^2$.
  \[
  S = \{ \text{Ужасная формула пугающая студентов}\}  
  \]
  \item По имеющимся данным рассчитываем значение $S_{obs}$.
  \item Рассчитываваем $P$-значение — вероятность $\P(S > S_{obs})$.
  \item Если $P$-значение мало, то отвергаем гипотезу $H_0$.
\end{enumerate}

\end{frame}

\begin{frame}{А что если\ldots}

  \begin{enumerate}[<+->]
    \item \alert{Идеальные условия} нарушены.
    \item Наблюдений \alert{не достаточно}, чтобы считать $S \sim \chi^2$.
    \item Подходящей \alert{теоремы нет}. 
  \end{enumerate}
  
  \pause
  Вместо $\chi^2$ распределения нужно использовать \alert{верное} распределение статистики $S$.

\end{frame}


\begin{frame}{Идея бутстрэпа}

  При больших $n$ можно \alert{оценить закон распределения} статистики $S$!

  \pause

  И вместо обещанного теоремой $\chi^2$-распределения \alert{использовать оценку распределения}.

  \pause

  \begin{block}{Предупреждение}
  Бутстрэп является асимптотическим методом и формально требует $n \to \infty$. 
  \end{block}

  \pause 

  Часто оказывается, что для хорошей оценки закона распределения $S$ нужно меньшее $n$, чем для теоремы с идеальными условиями.
\end{frame}





\begin{frame}{Наивный бутстрэп}

\begin{block}{Доверительный интервал для медианы}
Есть случайная выборка $y_1$, \ldots, $y_n$ из непрерывного распределения, $n$ велико. 

Посчитали выборочную медиану $\hat m$. Хотим построить доверительный интервал для медианы $m$. 
\end{block}

\pause
\begin{enumerate}[<+->]
  \item Из исходной выборки $y_1$, \ldots, $y_n$ построим бутстрэп-выборку $y^*_1$, \ldots, $y^*_n$:
  
  Выберем случайно $n$ наблюдений с повторениями. 
  \item На базе бутстрэп-выборки посчитаем очередную выборочную медиану $\hat m_j^*$.
  \item Повторим первые два шага много раз: $j = 1, \ldots, 10000$.
\end{enumerate}


\end{frame}


\begin{frame}{Наивный бутстрэп: формула интервала}

Хотим доверительный интервал для истинной медианы $m$ и уже раздобыли 10000 бустрэп выборочных медиан $\hat m^*_1$, \ldots, $\hat m^*_{10000}$.

\pause
\begin{block}{Доверительный интервал}

  \[
  \qL(\hat m^*) \leq m \leq \qR(\hat m^*),  
  \]

  где $\qL(\hat m^*)$ и $\qR(\hat m^*)$ — нужные левый и правый квантили. 
\end{block}

\pause
\begin{block}{Хочу 95\% доверительный интервал}

\begin{enumerate}[<+->]
  \item Отбрасываю 2.5\% самых маленьких $\hat m^*_j$ и 
  2.5\% самых больших $\hat m^*_j$.
  \item Крайние значения оставшихся $\hat m^*_j$ и будут границами интервала. 
\end{enumerate}
\end{block}

\end{frame}

\begin{frame}{Бутстрэп: проверка гипотез}

\pause
\alert{Скалярный параметр}
\begin{enumerate}[<+->]
  \item Гипотеза $H_0: \beta_x = 42$ против $\beta_x \neq 42$.
  \item Проверяем, входит ли $42$ в доверительный интервал. 
\end{enumerate}


\end{frame}
  

