\documentclass[14pt,xcolor=dvipsnames]{beamer}


\input{preamble.tex}
\usepackage{physics}



% tikz block

\usepackage{pgfplots}
\pgfplotsset{compat=newest}

\usepackage{tikz}
\usetikzlibrary{calc}
\usetikzlibrary{quotes,angles}
\usetikzlibrary{arrows}
\usetikzlibrary{arrows.meta}
\usetikzlibrary{positioning,intersections,decorations.markings}
\usetikzlibrary{patterns}

\usepackage{tkz-euclide} 
%\tikzset{>=latex}

\tikzset{cross/.style={cross out, draw=black, minimum size=2*(#1-\pgflinewidth), inner sep=0pt, outer sep=0pt},
%default radius will be 1pt. 
cross/.default={5pt}}

\colorlet{veca}{red}
\colorlet{vecb}{blue}
\colorlet{vecc}{olive}


\newcommand{\grid}{\draw[color=gray,step=1.0,dotted] (-2.1,-2.1) grid (9.6,6.1)}

% end tikz block

\newcommand{\R}{\mathbb{R}}
\newcommand{\Rot}{\mathrm{R}}
\newcommand{\HH}{\mathrm{H}}
\newcommand{\Id}{\mathrm{I}}
\newcommand{\RR}{\mathbb{R}}
\newcommand{\ZZ}{\mathbb{Z}}
\newcommand{\la}{\lambda}


\newcommand{\ba}{\mathbf{a}}
\newcommand{\be}{\mathbf{e}}
\newcommand{\bb}{\mathbf{b}}
\newcommand{\bc}{\mathbf{c}}
\newcommand{\bd}{\mathbf{d}}
\newcommand{\bx}{\mathbf{x}}
\newcommand{\bff}{\mathbf{f}} % \bf is already def
\newcommand{\bv}{\mathbf{v}}
\newcommand{\bzero}{\mathbf{0}}


\DeclareMathOperator{\Lin}{Span}
\DeclareMathOperator{\Span}{Span}
\DeclareMathOperator{\Image}{Image}
\DeclareMathOperator{\LL}{L}







\begin{document}


\begin{frame} % название лекции


\lecturetitle{Бутстрэп}

\end{frame}


\input{chapters/lec_020_fra_010_linear_combination.tex}

\input{chapters/lec_020_fra_020_linear_span.tex}

\input{chapters/lec_020_fra_030_vector_space.tex}

\input{chapters/lec_020_fra_040_matrix.tex}


\input{chapters/lec_020_fra_050_rank.tex}


\begin{frame}

\lecturetitle{Умножение матрицы на вектор}

\todo{Это видеофрагмент с доской, слайдов здесь нет :)}

\end{frame}



\begin{frame}
\lecturetitle{Умножение матрицы на матрицу}
\todo{Это видеофрагмент с доской, слайдов здесь нет :)}
\end{frame}
    


\begin{frame}
\lecturetitle{Три взгляда на умножение матриц}
\todo{Это видеофрагмент с доской, слайдов здесь нет :)}
\end{frame}
    

\begin{frame}
\lecturetitle{Решение системы уравнений методом Гаусса}
\todo{Это видеофрагмент с доской, слайдов здесь нет :)}
\end{frame}
    

\input{chapters/lec_020_fra_100_gauss_properties.tex}


\begin{frame}
\lecturetitle{Задача о шахматной доске}
\todo{Это видеофрагмент с доской, слайдов здесь нет :)}
\end{frame}


\end{document}
