% !TEX root = ../om_metrics_14.tex

\begin{frame} % название фрагмента

\videotitle{Мэтчинг}

\end{frame}



\begin{frame}{Мэтчинг: план}
  \begin{itemize}[<+->]
    \item Задача оценки эффекта влияния.
    \item Алгоритм мэтчинга.
    \item Мера склонности к воздействию. 
  \end{itemize}

\end{frame}


\begin{frame}{Цель анализа}

Данные:

$y_i \in \R$ — значение целевой переменной;

$a_i \in \{0, 1\}$ — индикатор того, что индивид $i$ получил воздействие;

$x_i \in \R$ — прочие характеристики индивида.

\pause

Хотим оценить \alert{влияние} бинарной переменной $a_i$ на $y_i$.

\end{frame}

\begin{frame}{Формализуем понятие «влияет»}

\alert{Гипотетические} значения:

$y_i(0)$ — значение целевой переменной, \alert{если бы} индивид не получил воздействие;

$y_i(1)$ — значение целевой переменной, \alert{если бы} индивид получил воздействие;

\pause 


\begin{block}{Задача}

Хотим оценить \alert{средний эффект воздействия}:

\[
ATE  = E(y_i(1) - y_i(0)).
\]
\end{block}


\end{frame}



\begin{frame}{А в чём проблема?}

\begin{block}{Задача}

    Хотим оценить \alert{средний эффект воздействия}:
    
    \[
    ATE  = E(y_i(1) - y_i(0)).
    \]
\end{block}

\pause

Наблюдаем \alert{только одну} из гипотетических величин!

У тех, кто получил воздействие, видим $y_i(1)$.

У тех, кто не получил воздействие, видим $y_i(0)$.

\[
y_i = y_i(a_i).  
\]


\end{frame}



\begin{frame}{Рандомизированный эксперимент}


\begin{block}{Рандомизированный эксперимент}
Воздействие $a_i$ назначается или выбирается случайно и не зависит от $y_i(0)$ и $y_i(1)$.

\[
  a_i \perp y_i(0), y_i(1)
\]

\end{block}

\pause
При назначении воздействия мы \alert{не предпочитаем} тех, кто на на него лучше среагирует. 


\alert{Нет самостоятельного выбора} воздействия индивидами, умеющими прогнозировать $y_i(0)$ и $y_i(1)$.

\end{frame}



\begin{frame}
  \frametitle{В мире розовых пони\ldots}

  В прекрасном мире \alert{рандомизированного эксперимента}:

  \alert{Любая} регрессия даст несмещённую и состоятельную оценку $ATE$.
  \pause
  \[
  \hat y_i = \hat \beta_1 + \hat \beta_a a_i,  
  \]
  \pause
  \[
  \hat y_i = \hat \beta_1 + \hat \beta_a a_i + \hat \beta_x x_i,  
  \]
  \pause
  В рандомизированном эксперименте:
  \[
  \plim_{n\to\infty} \hat \beta_a = ATE, \quad \E(\hat\beta_a) = ATE.  
  \]


\end{frame}

\begin{frame}{Отсутствие рандомизации}

  $y_i$ — результат теста по теории вероятностей;

  $a_i \in \{0, 1\}$ — пользовался ли студент шпаргалкой;

  $x_i$ — уровень знаний студента. 
  
  \pause
  Предположим, что только слабые студенты пользуются шпаргалкой. 
  \pause
  \[
    \hat y_i = \hat \beta_1 + \hat \beta_a a_i
  \]
  Регрессия недооценивает эффект, $E(\hat \beta_a) < ATE$.
\end{frame}


\begin{frame}{Отсутствие рандомизации}

  $y_i$ — результат теста по теории вероятностей;

  $a_i \in \{0, 1\}$ — пользовался ли студент продвинутым учебником;

  $x_i$ — уровень знаний студента. 
  
  \pause
  Предположим, что только сильные студенты пользуются продвинутым учебником. 
  \pause
  \[
    \hat y_i = \hat \beta_1 + \hat \beta_a a_i
  \]
  Регрессия переоценивает эффект, $E(\hat \beta_a) > ATE$.
\end{frame}



\begin{frame}
  \frametitle{Суровая реальность\ldots}

  Узнав о возможности получить воздействие, \alert{индивиды сами решают}, получать ли воздействие. 
  
  
  Значение $a_i$ \alert{связано с} $y_i(0)$ и $y_i(1)$. 

  \pause 

  В общем случае всё пропало, найти состоятельные и несмещённые оценки $ATE$ невозможно. 

  \pause 

  \begin{block}{Последняя надежда!}
  Воздействие не зависит от гипотетических исходов при фиксированном $x_i$.
  \[
  a_i \perp y_i(0), y_i(1) \mid  x_i
  \]
  \end{block}

\end{frame}


\begin{frame}{Мэтчинг и регрессия}

  \begin{enumerate}[<+->]
    \item Создаём кластеры исходных наблюдений. 
    
    В один кластер входят наблюдения с \alert{близкими} характеристиками $x_i$
    и разными значениями $a_i$.

    Внутри кластера $a_i$ не зависит от потенциальных исходов $y_i(0)$ и $y_i(1)$.

    \item Строим регрессию $y_i$ на $a_i$ с использованием \alert{весов} наблюдений 
    и с \alert{робастными стандартными ошибками}.

    Веса нивелируют различие в $x_i$ для получивших и не получивших воздействие. 

    Робастные стандартные ошибки учтут гетероскедастичность и 
    корреляцию внутри кластера. 
  \end{enumerate}




\end{frame}


\begin{frame}{Создаём кластеры наблюдений!}

  Кластеры бывают двух типов:
  \begin{itemize}[<+->]
    \item Один индивид, получивший воздействие, и несколько \alert{близких к нему} индвидов, не получивших воздействия. 
    \item Один индивид, не получивший воздействия, и несколько \alert{близких к нему} индвидов, получивших воздействие. 
  \end{itemize}

  При создании кластеров наблюдения получат веса.

  \pause
  Как выбрать близких индивидов к данному?
  \begin{itemize}[<+->]
    \item \alert{Расстояние Махаланобиса} для прочих характеристик $x_i$.
    \item \alert{Мера склонности} к воздействию (propensity score), $\widehat{ps}(x_i)$.
  \end{itemize}

\end{frame}

\begin{frame}{Мера склонности}

\begin{block}{Мера склонности}
Оценка вероятности получить воздействие, полученная с помощью прочих характеристик индивида $x_i$.
\[
\widehat{ps}(x_i) = \hat P(a_i = 1 \mid x_i).
\]
\end{block}
\pause
Например, можно оценить \alert{логистическую регрессию}
\[
  \hat P(a_i = 1 \mid x_i) = \Lambda(\hat\beta_1 + \hat\beta_x x_i),
\]
где $\Lambda(t) = \exp(t)/ (1 + \exp (t))$.

\end{frame}


\begin{frame}{Мера склонности и веса}
  \begin{block}{Мера склонности}
    Оценка вероятности получить воздействие, полученная с помощью прочих характеристик индивида $x_i$.
    \[
    \widehat{ps}(x_i) = \hat P(a_i = 1 \mid x_i).
    \]
  \end{block}
  \pause
  \begin{block}{Вес наблюдения}
    \[
      w_i = \begin{cases}
          1 / \widehat{ps}(x_i), \text{ если } a_i = 1; \\
          1 / (1 - \widehat{ps}(x_i)), \text{ если } a_i = 0; \\
      \end{cases}
    \]
  \end{block}
  \pause
  Вес наблюдения — насколько редко встречаются наблюдения с данным $a_i$ среди похожих по $x_i$. 
\end{frame}

\begin{frame}{Рецепт и большая наука}

\begin{block}{Краткий рецепт}
\begin{enumerate}
  \item Создаём кластеры наблюдений отличающихся только воздействием. 
  \item Оцениваем регрессию и получаем доверительный интервал для эффекта влияния.
\end{enumerate}
\end{block}

\alert{Важно:} наличие воздействия $a_i$ должно не зависеть 
от потенциальных исходов $y_i(1)$ и $y_i(0)$ при фиксированных $x_i$.

\pause

\begin{block}{Много интересного!}
  \begin{enumerate}
    \item ATE, ATT, ATM, непараметрические методы, \ldots
    \item Многие методы ещё не имеют глубокого теоретического обоснования. 
  \end{enumerate}
\end{block}
  
\end{frame}

\begin{frame}{Соседи и мэтчинг: итоги}

  \begin{block}{Общая идея}
    Метод ближайших соседей: используем похожие наблюдения, чтобы \alert{прогнозировать} $y_i$.
    
    Мэтчинг: используем похожие наблюдения, чтобы \alert{оценить эффект} $a_i$.
    \end{block}

\pause
    
  \begin{itemize}[<+->]
    \item Мэтчинг требует предположения $a_i \perp y_i(1), y_i(0) \mid x_i$.
    \item Мэтчинг — область активных исследований.
  \end{itemize}

\end{frame}

\end{document}

