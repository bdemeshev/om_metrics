\documentclass[12pt]{article}

%\usepackage{tikz} % картинки в tikz
\usepackage{microtype} % свешивание пунктуации

% \usepackage{listings}
\usepackage{minted,tikz}

\usepackage{array} % для столбцов фиксированной ширины

\usepackage{indentfirst} % отступ в первом параграфе

\usepackage{sectsty} % для центрирования названий частей
\allsectionsfont{\centering}

\usepackage{amsmath, amssymb, amsthm} % куча стандартных математических плюшек

\usepackage{amsfonts}

\usepackage{comment}

\usepackage[top=2cm, left=1.2cm, right=1.2cm, bottom=2cm]{geometry} % размер текста на странице

\usepackage{lastpage} % чтобы узнать номер последней страницы

\usepackage{enumitem} % дополнительные плюшки для списков
%  например \begin{enumerate}[resume] позволяет продолжить нумерацию в новом списке
\usepackage{caption}

\usepackage{physics}

\usepackage{hyperref} % гиперссылки

\usepackage{multicol} % текст в несколько столбцов


\usepackage{fancyhdr} % весёлые колонтитулы
\pagestyle{fancy}
\lhead{Промежуточный экзамен, осень 2021}
\chead{}
\rhead{}
\lfoot{}
\cfoot{}
\rfoot{}
\renewcommand{\headrulewidth}{0.4pt}
\renewcommand{\footrulewidth}{0.4pt}

\let\P\relax
\DeclareMathOperator{\P}{\mathbb{P}}
\DeclareMathOperator{\Cov}{\mathbb{C}ov}
\DeclareMathOperator{\E}{\mathbb{E}}
\DeclareMathOperator{\Var}{\mathbb{V}ar}
\newcommand{\cN}{\mathcal{N}}

\usepackage{todonotes} % для вставки в документ заметок о том, что осталось сделать
% \todo{Здесь надо коэффициенты исправить}
% \missingfigure{Здесь будет Последний день Помпеи}
% \listoftodos - печатает все поставленные \todo'шки


% более красивые таблицы
\usepackage{booktabs}
% заповеди из докупентации:
% 1. Не используйте вертикальные линни
% 2. Не используйте двойные линии
% 3. Единицы измерения - в шапку таблицы
% 4. Не сокращайте .1 вместо 0.1
% 5. Повторяющееся значение повторяйте, а не говорите "то же"

\usepackage{tikz, pgfplots} % язык для рисования графики из latex'a
\usetikzlibrary{trees} % прибамбас в нем для рисовки деревьев
\usetikzlibrary{arrows} % прибамбас в нем для рисовки стрелочек подлиннее
\usepackage{tikz-qtree} % прибамбас в нем для рисовки деревьев
\usetikzlibrary{automata, arrows, positioning, calc}


\usepackage{fontspec}
\usepackage{polyglossia}

\setmainlanguage{english}
\setotherlanguages{russian}

% download "Linux Libertine" fonts:
% http://www.linuxlibertine.org/index.php?id=91&L=1
\setmainfont{Linux Libertine O} % or Helvetica, Arial, Cambria

\setmonofont{Liberation Mono} % minted использует моноширинный шрифт, нужен моноширинный шрифт с кириллицей
% why do we need \newfontfamily:
% http://tex.stackexchange.com/questions/91507/
\newfontfamily{\cyrillicfonttt}{Linux Libertine O}
\newfontfamily{\cyrillicfont}{Linux Libertine O}
\newfontfamily{\cyrillicfontsf}{Linux Libertine O}


\AddEnumerateCounter{\asbuk}{\russian@alph}{щ} % для списков с русскими буквами
% \setlist[enumerate, 2]{label=\asbuk*),ref=\asbuk*}




\begin{document}

\begin{enumerate}
\item 
\item 
\item 

\item 
\begin{enumerate}
    \item $\hat x_{101} = 174.9010 - 0.1687 \cdot 158 = 148.25$.
    \item Обозначим для удобства буквой $a$ иксы с первого по 99-й,
    а буквой $b$ — со второго по 100-й. 

    Находим $\bar a = (100\bar x - x_{100})/99 = 149.68$, $\bar b = (100\bar x - x_{1})/99=149.66$.
    Эти средние очень близки и это не случайно. 
    Мы усредняем по большому количеству наблюдений, поэтому замена первого наблюдения 
    на сотое мало что меняет.
    
    В условии у нас есть регрессия $\hat b_i = 174.901 - 0.1687 a_i$.
    \[
    \frac{\sum (a_i - \bar a)(b_i - \bar b)}{\sum (a_i - \bar a)^2} = -0.1687    
    \]

    Нас интересует коэффициент наклона в регрессии $\hat a_i = \hat\gamma_1 +\hat\gamma_2 b_i$.
    \[
    \hat\gamma_2 = \frac{\sum (a_i - \bar a)(b_i - \bar b)}{\sum (b_i - \bar b)^2}     
    \]

    Замечаем, что между этими двумя формулами при таком количестве слагаемых практически нет разницы.
    Числители в точности одинаковые, а знаменатели отличаются незначительно. 

    Близость знаменателей нагляднее:
    \[
    \sum (a_i - \bar a)^2 = \sum a_i^2 - n \bar a^2 \approx \sum b_i^2 - n \bar b^2.
    \]

    
    Поэтому обратная регрессия в данном случае примерно описывается тем же (!!), а не обратным соотношением, 
    как кажется на первый взгляд.

    \item $\hat x_{0} \approx 174.9010 - 0.1687 \cdot 160 = 147.91$.


\end{enumerate}

\item 
\item *

\end{enumerate}


\end{document}
