% arara: xelatex: {shell: yes}
\documentclass[12pt]{article}

%\usepackage{tikz} % картинки в tikz
\usepackage{microtype} % свешивание пунктуации

% \usepackage{listings}
\usepackage{minted,tikz}

\usepackage{array} % для столбцов фиксированной ширины

\usepackage{indentfirst} % отступ в первом параграфе

\usepackage{sectsty} % для центрирования названий частей
\allsectionsfont{\centering}

\usepackage{amsmath, amssymb, amsthm} % куча стандартных математических плюшек

\usepackage{amsfonts}

\usepackage{comment}

\usepackage[top=2cm, left=1.2cm, right=1.2cm, bottom=2cm]{geometry} % размер текста на странице

\usepackage{lastpage} % чтобы узнать номер последней страницы

\usepackage{enumitem} % дополнительные плюшки для списков
%  например \begin{enumerate}[resume] позволяет продолжить нумерацию в новом списке
\usepackage{caption}

\usepackage{physics}

\usepackage{hyperref} % гиперссылки

\usepackage{multicol} % текст в несколько столбцов


\usepackage{fancyhdr} % весёлые колонтитулы
\pagestyle{fancy}
\lhead{Промежуточный экзамен, осень 2021, пересдача}
\chead{}
\rhead{}
\lfoot{}
\cfoot{}
\rfoot{}
\renewcommand{\headrulewidth}{0.4pt}
\renewcommand{\footrulewidth}{0.4pt}

\let\P\relax
\DeclareMathOperator{\P}{\mathbb{P}}
\DeclareMathOperator{\Cov}{\mathbb{C}ov}
\DeclareMathOperator{\E}{\mathbb{E}}
\DeclareMathOperator{\Var}{\mathbb{V}ar}
\newcommand{\cN}{\mathcal{N}}

\usepackage{todonotes} % для вставки в документ заметок о том, что осталось сделать
% \todo{Здесь надо коэффициенты исправить}
% \missingfigure{Здесь будет Последний день Помпеи}
% \listoftodos - печатает все поставленные \todo'шки


% более красивые таблицы
\usepackage{booktabs}
% заповеди из докупентации:
% 1. Не используйте вертикальные линни
% 2. Не используйте двойные линии
% 3. Единицы измерения - в шапку таблицы
% 4. Не сокращайте .1 вместо 0.1
% 5. Повторяющееся значение повторяйте, а не говорите "то же"

\usepackage{tikz, pgfplots} % язык для рисования графики из latex'a
\usetikzlibrary{trees} % прибамбас в нем для рисовки деревьев
\usetikzlibrary{arrows} % прибамбас в нем для рисовки стрелочек подлиннее
\usepackage{tikz-qtree} % прибамбас в нем для рисовки деревьев
\usetikzlibrary{automata, arrows, positioning, calc}


\usepackage{fontspec}
\usepackage{polyglossia}

\setmainlanguage{english}
\setotherlanguages{russian}

% download "Linux Libertine" fonts:
% http://www.linuxlibertine.org/index.php?id=91&L=1
\setmainfont{Linux Libertine O} % or Helvetica, Arial, Cambria

\setmonofont{Liberation Mono} % minted использует моноширинный шрифт, нужен моноширинный шрифт с кириллицей
% why do we need \newfontfamily:
% http://tex.stackexchange.com/questions/91507/
\newfontfamily{\cyrillicfonttt}{Linux Libertine O}
\newfontfamily{\cyrillicfont}{Linux Libertine O}
\newfontfamily{\cyrillicfontsf}{Linux Libertine O}


\AddEnumerateCounter{\asbuk}{\russian@alph}{щ} % для списков с русскими буквами
% \setlist[enumerate, 2]{label=\asbuk*),ref=\asbuk*}




\begin{document}

\section*{Формальности}

На экзамен выделено 120 минут времени, плюс дополнительные 5 минут на загрузку работы. 
Можно пользоваться лекциями и любыми источниками, общаться с другими лицами во 
время экзамена нельзя. Экзамен проводится без прокторинга.
Ответ без решения не засчитывается.  

Более сложные задачи отмечены звёздочкой в начале условия. 

\section*{Как сдавать ответ на «ручные» вопросы?}

Написать текст на листочках и прикрепить фотографии.
Также можно писать текст на планшете и прикрепить готовый pdf.


\section*{Как сдавать ответ на компьютерные вопросы?}

Скопировать предложенный шаблон ответа в Rstudio.
Ниже каждого пункта привести код, требующийся для решения пункта задачи. 
После кода нужно привести чёткий ответ на поставленный вопрос в виде комментария после \verb|#|.

Пример условия шаблона:

\begin{minted}{r}
    # а) Найдите косинус числа 42.

    # ...

    # Косинус равен ...
\end{minted}    

Пример ответа:

\begin{minted}{r}
    # а) Найдите косинус числа 42.

    cos(42)
    
    # Косинус равен -0.4.
\end{minted}

Затем полученный файл .R надо сохранить и прикрепить на платформе. 
Один файл .R на всю пятую задачу и один — на шестую. 


\section*{Ни пуха, ни пера!}

\newpage
\begin{enumerate}
    \item Исследователь Винни-Пух оценил модель множественной регрессии.
    \[
    \hat y_i = \hat\beta_1 + \hat\beta_2 x_i + \hat\beta_3 z_i.
    \]

    В качестве $\hat\beta_2$ возьмите количество букв в своей фамилии,
    в качестве $se(\hat\beta_2)$ — количество букв в своём имени. 

    У Винни-Пуха 1000 наблюдений и он считает, что выполнены классические предпосылки 
    на ошибки модели, $\E(u_i \mid X) = 0$, $\Var(u_i \mid X) = \sigma^2$, 
    $\Cov(u_i, u_j \mid X) = 0$.


    \begin{enumerate}
        \item Постройте 90\%-й доверительный интервал для $\beta_2$.
        \item Проверьте гипотезу $H_0: \beta_2 = 3$ против альтернативной 
        $\beta_2 > 3$ на уровне значимости 10\%.
    \end{enumerate}

    \item Заполните пропуски Q1-Q9, пояснив формулами как конкретно, и в каком порядке они заполнялись.
    
    \begin{tabular}{lr} \toprule
    Показатель & Значение \\
    \midrule
    Выборочная корреляция прогнозов и зависимой переменной  & Q1 \\
    $R^2$     			& Q2 \\
    Оценка дисперсии ошибки 		& Q3 \\
    Число наблюдений		& 800 \\
    RSS & 37 \\
    ESS & 42.9 \\
    TSS & Q5 \\
    \bottomrule
    \end{tabular}


    \begin{tabular}{rrrrrrr}
    \toprule
                 & Coef. 	& St. error	& t-stat & P-value	& Lower 95\% 	& Upper 95\% \\
    \midrule
    Intercept 	& -25.24 	& 2.0 	& Q6 		& 0 	&  Q7		& -21.31 \\
    totspan		& 1.7		& Q8    & 30.4 	    & 0 	&  Q9	    & Q4 \\
    \bottomrule
    \end{tabular}
    
\newpage
    \item (*) Регрессию оценили с помощью МНК по 50 наблюдениям:
    \[
    \hat y_i = -17.3 + 3.92 x_i - 0.63 z_i.    
    \]

    В исходной модели $y_i = \beta_1 + \beta_2 x_i + \beta_3 z_i + u_i$ ошибки удовлетворяют 
    предпосылкам теоремы Гаусса-Маркова и нормально распределены.


    Матрица регрессоров $X$ с первым столбцом из единиц такова, что 
    \[
    (X^TX)^{-1} = \begin{pmatrix}
        0.196 & -0.0114 & -0.008 \\
        -0.0114 &  0.00074 & 0.00041 \\
        -0.008 & 0.00041 & 0.0208 \\
    \end{pmatrix}.    
    \]

    Сумма квадратов остатков модели равна 11334.

\begin{enumerate}
    \item Оцените дисперсию случайной ошибки.
    \item Найдите стандартную ошибку для $\hat\beta_3$.
    \item На уровне значимости 10\% протестируйте значимость оценки $\hat\beta_3$.
    \item На уровне значимости 10\% протестируйте гипотезу $H_0$: $\beta_2 = 2 \beta_3$.
\end{enumerate}


\item (*) Придумайте и явно выпишите набор данных, для которого одновременно выполняются три свойства. 
Регрессия $y_i$ на $x_i$ по первой половине данных даёт положительную оценку коэффициента при $x_i$.
Регрессия $y_i$ на $x_i$ по второй половине данных даёт положительную оценку коэффициента при $x_i$.
Регрессия $y_i$ на $x_i$ по всему набору даёт отрицательную оценку коэффициента при $x_i$.


\newpage
    \item Мы будем работать с набором данных \verb|LifeCycleSavings|.

    Шаблон с вопросами и пропусками для вашего кода и ответов приведён ниже. 

    \begin{minted}[]{R}
        library(tidyverse)
        library(skimr)

        d = LifeCycleSavings

        # а) Отберите наблюдения с 
        # подушевым доходом dpi более медианного.
        # Сколько наблюдений в полученном наборе данных?

        # ...

        # В наборе данных ... наблюдений.

        # Далее работаем с полученным набором данных.
        # б) Чему равно максимальное значение переменной sr?
        # (в данных sr измерено в процентах от располагаемого дохода)

        # ...

        # Максимальное значение sr равно ... 

        # в) Постройте регрессию размера личных сбережений sr на 
        # константу, долю населения младше 15 лет pop15, 
        # долю населения старше 75 pop75.
        # Выпишите полученное уравнение регрессии. 

        # ...

        # hat_sr_i = ... + ... + ... 

        # г) Постройте 99%-й доверительный интервал для коэффициента 
        # при pop75. 

        # ...

        # Интервал: [..., ...]

        # д) Какой прогноз личных сбережений для страны 
        # с 20% населения младше 15 лет и 10% населения старше 75 лет
        # даёт данная модель?

        # ...

        # Прогноз равен ...
    \end{minted}

    \newpage
\item (*) Вернёмся к набору данных \verb|LifeCycleSavings|.

Шаблон с вопросами и пропусками для вашего кода и ответов приведён ниже. 

\begin{minted}[]{R}
    library(tidyverse)
    library(skimr)

    d = LifeCycleSavings

    # а) Для переменной sr посчитайте логарифм. 
    # В качестве ответа приведите наибольшее значение 
    # логарифма.

    # ...

    # Наибольшее значение логарифма равно ... 

    # б) Постройте регрессию логарифма размера личных сбережений log(sr) на 
    # константу, pop15, pop75 и их произведение.
    # Выпишите полученное уравнение регрессии. 

    # ...

    # log(\hat sr)_i = ... + ... + ... + ...

    # в) Найдите оценку ковариационной матрицы оценок коэффициентов
    # В качестве ответа выпишите  
    # оценку ковариации коэффициентов при pop15 и pop75.

    # ... 

    # Оценка ковариации равна ...

    # г) Постройте 99%-й доверительный интервал для 
    # разницы коэффициентов при pop15 и pop75.

    # ...

    # Интервал: [..., ...]

    # д) Какой прогноз личных сбережений (не логарифма!) для страны 
    # с 30% населения младше 15 лет и 3% населения старше 75 лет 
    # даёт данная модель?
    
    # ...

    # Прогноз равен ...
\end{minted}



\end{enumerate}


\end{document}
