% arara: xelatex: {shell: yes}
\documentclass[12pt]{article}

%\usepackage{tikz} % картинки в tikz
\usepackage{microtype} % свешивание пунктуации

% \usepackage{listings}
\usepackage{minted}
\usepackage{tikz}

\usepackage{array} % для столбцов фиксированной ширины

\usepackage{indentfirst} % отступ в первом параграфе

\usepackage{sectsty} % для центрирования названий частей
\allsectionsfont{\centering}

\usepackage{amsmath, amssymb, amsthm} % куча стандартных математических плюшек

\usepackage{amsfonts}

\usepackage{comment}

\usepackage[top=2cm, left=1.2cm, right=1.2cm, bottom=2cm]{geometry} % размер текста на странице

\usepackage{lastpage} % чтобы узнать номер последней страницы

\usepackage{enumitem} % дополнительные плюшки для списков
%  например \begin{enumerate}[resume] позволяет продолжить нумерацию в новом списке
\usepackage{caption}

\usepackage{physics}

\usepackage{hyperref} % гиперссылки

\usepackage{multicol} % текст в несколько столбцов


\usepackage{fancyhdr} % весёлые колонтитулы
\pagestyle{fancy}
\lhead{Финальный экзамен, зима 2021}
\chead{}
\rhead{}
\lfoot{}
\cfoot{}
\rfoot{}
\renewcommand{\headrulewidth}{0.4pt}
\renewcommand{\footrulewidth}{0.4pt}

\let\P\relax
\DeclareMathOperator{\P}{\mathbb{P}}
\DeclareMathOperator{\Cov}{\mathbb{C}ov}
\DeclareMathOperator{\E}{\mathbb{E}}
\DeclareMathOperator{\Var}{\mathbb{V}ar}
\newcommand{\cN}{\mathcal{N}}

\usepackage{todonotes} % для вставки в документ заметок о том, что осталось сделать
% \todo{Здесь надо коэффициенты исправить}
% \missingfigure{Здесь будет Последний день Помпеи}
% \listoftodos - печатает все поставленные \todo'шки


% более красивые таблицы
\usepackage{booktabs}
% заповеди из докупентации:
% 1. Не используйте вертикальные линни
% 2. Не используйте двойные линии
% 3. Единицы измерения - в шапку таблицы
% 4. Не сокращайте .1 вместо 0.1
% 5. Повторяющееся значение повторяйте, а не говорите "то же"

\usepackage{tikz, pgfplots} % язык для рисования графики из latex'a
\usetikzlibrary{trees} % прибамбас в нем для рисовки деревьев
\usetikzlibrary{arrows} % прибамбас в нем для рисовки стрелочек подлиннее
\usepackage{tikz-qtree} % прибамбас в нем для рисовки деревьев
\usetikzlibrary{automata, arrows, positioning, calc}


\usepackage{fontspec}
\usepackage{polyglossia}

\setmainlanguage{english}
\setotherlanguages{russian}

% download "Linux Libertine" fonts:
% http://www.linuxlibertine.org/index.php?id=91&L=1
\setmainfont{Linux Libertine O} % or Helvetica, Arial, Cambria

\setmonofont{Liberation Mono} % minted использует моноширинный шрифт, нужен моноширинный шрифт с кириллицей
% why do we need \newfontfamily:
% http://tex.stackexchange.com/questions/91507/
\newfontfamily{\cyrillicfonttt}{Linux Libertine O}
\newfontfamily{\cyrillicfont}{Linux Libertine O}
\newfontfamily{\cyrillicfontsf}{Linux Libertine O}


\AddEnumerateCounter{\asbuk}{\russian@alph}{щ} % для списков с русскими буквами
% \setlist[enumerate, 2]{label=\asbuk*),ref=\asbuk*}




\begin{document}

\section*{Формальности}

На экзамен выделено 120 минут времени, плюс дополнительные 5 минут на загрузку работы. 
Можно пользоваться лекциями и любыми источниками, общаться с другими лицами во 
время экзамена нельзя. Экзамен проводится без прокторинга.
Ответ без решения не засчитывается.  

Более сложные задачи отмечены звёздочкой в начале условия. 

\section*{Как сдавать ответ на «ручные» вопросы?}

Написать текст на листочках и прикрепить фотографии.
Также можно писать текст на планшете и прикрепить готовый pdf.


\section*{Как сдавать ответ на компьютерные вопросы?}

Скопировать предложенный шаблон ответа в Rstudio.
Ниже каждого пункта привести код, требующийся для решения пункта задачи. 
После кода нужно привести чёткий ответ на поставленный вопрос в виде комментария после \verb|#|.

Пример условия шаблона:

\begin{minted}{r}
    # а) Найдите косинус числа 42.

    # ...

    # Косинус равен ...
\end{minted}    

Пример ответа:

\begin{minted}{r}
    # а) Найдите косинус числа 42.

    cos(42)
    
    # Косинус равен -0.4.
\end{minted}

Затем полученный файл .R надо сохранить и прикрепить на платформе. 
Один файл .R на всю пятую задачу и один — на шестую. 


\section*{Ни пуха, ни пера!}

\newpage
\begin{enumerate}
    \item Исследователь оценивает модель $y_i = \beta_1 + \beta_2 z_i + u_i$ зависимости 
    роста зелёных человечков $y_i$ от их зелёности $z_i$ по разным наборам данных:

    \begin{tabular}{cccc}
        \toprule 
        Набор данных & Уравнение & $n$ & $RSS$ \\
        \midrule
        $A$ & $\hat y_i = 5 + 2 z_i$ & 100 & $200$ \\
        $B$ & $\hat y_i = 3 + 4 z_i$ & 100 & $300$ \\
        $A$ и $B$ & $\hat y_i = 2 + 3 z_i$ & 200 & $600$ \\
        \bottomrule
    \end{tabular}

    \begin{enumerate}
        \item Проверьте гипотезу о том, что зависимость одинакова для наборов данных $A$ и $B$.
        
        \item Найдите оценки коэффициентов в регрессии 
        \[
        \hat y_i = \hat\beta_1 + \hat\beta_2 a_i + \hat\beta_3 z_i + \hat\beta_4 z_i a_i,
        \]
        где $a_i$ — дамми-переменная равная 1 для зелёных человечков из группы А.
        
    \end{enumerate}

    \item Пират Джо изучает зависимость веса $y_i$ ящика с золотыми монетами от количества золотых монет в ящике $x_i$. 
    Джо предполагает простую зависимость $y_i = \beta_1 + \beta_2 x_i + u_i$ с $\E(u_i \mid x_i) = 0$.

    Также Джо считает, что веса всех монет независимы и одинаково распределены. 
    \begin{enumerate}
        \item Какая зависимость $\Var(u_i \mid x_i)$ от $x_i$ следует из предположений Джо?
        \item Какую регрессию стоит оценить Джо, чтобы получить эффективные оценки коэффициентов?
        \item Как Джо проинтерпретировать оценки $\hat\beta_1$ и $\hat\beta_2$?
    \end{enumerate}
    
    
\newpage
    \item (*) Исследователь Винни-Пух изучает зависимость бинарной переменной правильности мёда $y_i$ 
    от бинарной переменной правильности пчёл $x_i$ по таблице сопряжённости:

    \begin{tabular}{ccc}
        \toprule 
         & Правильные пчёлы & Неправильные пчёлы \\
        \midrule
        Правильный мёд & 50 & 70 \\
        Неправильный мёд & 100 & 80 \\
        \bottomrule
    \end{tabular}
    
    \begin{enumerate}
        \item Оцените линейную модель $y_i = \beta_1 + \beta_2 x_i + u_i$ с помощью линейной регрессии. 
        \item Оцените логистическую модель $\P(y_i = 1) = \Lambda(\beta_1 + \beta_2 x_i)$. 
        \item Постройте прогнозы вероятности правильности мёда у правильных пчёл по обеим моделям. 
    \end{enumerate}

\item (*) Широко известно, что оценка студента по эконометрике, $res_i$, зависит от числа выпитых чашек кофе, $c_i$:
\[
res_i = \beta_1 + \beta_2 cup_i + u_i.
\]
У преподавателя есть данные по $res_i$, однако величина $cup_i$ не наблюдаема. 
Преподаватель смог опросить опросить студентов и получил опросные данные по $cup_i^*$ —
заявленному числу выпитых чашек кофе, и по $cake_i^*$ — заявленному числу съеденных чиз-кейков. 

Других данных у исследователя нет. Исследователь предполагает, что $cup_i^* = cup_i + w_i$, и
$cake_i^* = \gamma_1 + \gamma_2 cup_i + \nu_i$. 


Наблюдения представляют собой случайную выборку. 
Величины $u_i$, $w_i$, $\nu_i$ и $cup_i$ независимы и имеют неизвестные дисперсии $\sigma^2_u$, $\sigma^2_w$,
$\sigma^2_{\nu}$, $\sigma^2_c$, величина $\gamma_2 \neq 0$ и так же неизвестна. 



\begin{enumerate}
    \item Найдите предел по вероятности МНК оценки $\hat\beta_2$ 
    в обычной регрессии $\widehat{res}_i = \hat\beta_1 + \hat\beta_2 cup_i^*$. 

    \item Предложите состоятельную оценку для неизвестного параметра $\beta_2$ и докажите её состоятельность. 
\end{enumerate}

\newpage
    \item Мы будем работать с набором данных по дефолтам индивидов \verb|Default|.

    Шаблон с вопросами и пропусками для вашего кода и ответов приведён ниже. 

    \begin{minted}[]{R}
        library(tidyverse)
        library(skimr)
        library(ISLR)

        d = Default

        # а) Оцените логистическую регрессию дефолта индивида default 
        # от среднего баланса на счету balance и дохода income. 

        # ...

        # Оценка зависимости имеет вид: ... = ... + ... + ...  

        # б) Постройте 90%-й доверительный интервал 
        # для среднего предельного эффекта для переменной income.

        # ...

        # Средний предельный эффект равен ...

        # в) Постройте прогноз вероятности дефолта для индивида 
        # с медианным доходом и медианным балансом. 

        # ...

        # Прогноз вероятности равен ...

        # г) Проверьте гипотезу о том, что модель идентична 
        # для студентов и не студентов против гипотезы о том,
        # что модель может отличаться всем коэффициентами. 

        # ...

        # Значение статистики ..., P-значение равно ... 
        # Статистический вывод ...

    \end{minted}

    \newpage
    \item Мы продолжаем работать с набором данных по дефолтам индивидов \verb|Default|.

    Шаблон с вопросами и пропусками для вашего кода и ответов приведён ниже. 

\begin{minted}[]{R}
    library(tidyverse)
    library(skimr)
    library(ISLR)

    d = Default

    # а) Оцените линейную модель зависимости дохода income от
    # баланса на карте balance и статуса студента student. 
    # Используйте робастные стандартные ошибки. 

    # ...

    # Оценка зависимости имеет вид: ... = ... + ... + ...

    # б) Постройте 90%-й робастный доверительный интервал для 
    # коэффициента при балансе. 

    # ...

    # Интервал имеет вид [...; ...]

    # в) Постройте 90%-й робастный доверительный интервал для ожидаемого 
    # дохода  для студента с медианным балансом на карточке. 

    # ... 

    # Интервал имеет вид [...; ...]

    # г) Проверьте гипотезу о том, что модель идентична для индивидов 
    # с дефолтом и без против гипотезы о том, что модель может 
    # отличаться всем коэффициентами. 

    # ...

    # Значение статистики ..., P-значение равно ...
    # Статистический вывод ...

    # д) Проверьте гипотезу об отсутствии гетероскедастичности с помощью 
    # теста Уайта. Используйте уровень значимости 10%.
    
    # ...

    # Значение статистики ..., P-значение равно ...
    # Статистический вывод ...
\end{minted}



\end{enumerate}


\end{document}
